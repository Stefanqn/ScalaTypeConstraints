\documentclass[11pt]{beamer}
%\hypersetup{hidelinks}
\usetheme{Montpellier}
%\usetheme{Pittsburgh}

\usepackage[utf8]{inputenc}
\usepackage{scala2}
\usepackage{amsmath}
\usepackage{amsfonts}
\usepackage{amssymb}
\usepackage{graphicx}
\usepackage{listings}
% --------------------------------------------------------------------------

% --------------------------------------------------------------------------

\author{Stefan}
\title{\textless:\textless}
%\setbeamercovered{transparent} 
%\setbeamertemplate{navigation symbols}{} 
%\logo{} 
%\institute{} 
\date{} 
\subject{sad-with-a-hat} 
\begin{document}


%% Title
\begin{frame}
\titlepage
Based on \href{http://blog.bruchez.name/2015/11/generalized-type-constraints-in-scala.html}{Generalized type constraints in Scala}
\end{frame}

%% ToC
\begin{frame} 
	\frametitle{Table of content} 
	\tableofcontents 
\end{frame}

\section{Introduction} 
\subsection{Usage}

\begin{frame}[fragile]
\frametitle{How is it useful?}
\begin{itemize}
\item on Option
\begin{lstlisting} 
def flatten[B](implicit ev: A <:< Option[B]): Option[B]
\end{lstlisting} %\pause

\item on Traversable
\begin{lstlisting} 
def toMap[K, V](implicit ev: A <:< (K, V)): Map[K, V]
\end{lstlisting} %\pause

\item on Try
\begin{lstlisting} 
def flatten[U](implicit ev: T <:< Try[U]): Try[U]
\end{lstlisting} 
\end{itemize}
\end{frame}

\begin{frame}{What do they have in common?}
\begin{itemize}
\item implicit parameter list, with a single parameter called ev
\item type of this parameter is of the form \lstinline{Type1 <:< Type2}
\end{itemize}
\end{frame}


\subsection{Meaning}
\begin{frame}[fragile]{Meaning}
\lstinline {Type1 <:< Type2} \newline\newline
\begin{block}{Meaning} %definition, theorem, lemma, proof, corollary, or example
\textit{Make sure that Type1 is a subtype of Type2, or else report an error.}
\end{block}
\end{frame}

\begin{frame}[fragile]

\begin{block}{Example 1} 
\begin{lstlisting} 
scala> val oo: Option[Option[Int]] = Some(Some(42))
oo: Option[Option[Int]] = Some(Some(42))

scala> oo.flatten
res1: Option[Int] = Some(42)
\end{lstlisting} 
\end{block}

\begin{block}{Example 2}
\begin{lstlisting} 
scala> val oi: Option[Int] = Some(42)
oi: Option[Int] = Some(42)

scala> oi.flatten
<console>:21: error: Cannot prove that Int <:< Option[B].
       oi.flatten
\end{lstlisting} 
\end{block} 
\end{frame}


\section{Questions} 
\subsection{Can’t we just use type bounds?}
\begin{frame}[fragile] {Plain example}
\begin{lstlisting} 
scala> def tuple[T, U](t: T, u: U) = (t, u)
scala> tuple("Lincoln", 42)
res1: (String, Int) = (Lincoln,42)
\end{lstlisting}
\end{frame}

\begin{frame}[fragile] {Example with upper bound}
\begin{lstlisting} 
scala> def tupleIfSubtype[T <: U, U](t: T, u: U) = (t, u)
scala> tupleIfSubtype("Lincoln", 42)
\end{lstlisting} 
\begin{uncoverenv}<2> 
\begin{lstlisting} 
res2: (String, Any) = (Lincoln,42)
\end{lstlisting} 
\end{uncoverenv}
\end{frame}

\begin{frame}[fragile] {Puzzler}
\begin{block}{Why?}
\textit{type inference solves a constraint system}
\end{block}\pause

\begin{block}{so given}
\begin{lstlisting} 
scala> def tupleIfSubtype[T <: U, U](t: T, u: U) = (t, u)
scala> tupleIfSubtype("Lincoln", 42)
res2: (String, Any) = (Lincoln,42)
\end{lstlisting} 
\end{block}

\begin{block}{the constraints are satisfied with}
\begin{itemize}
\item String is a String of course
\item Int is a subtype of Any
\item String is also a subtype of Any
\end{itemize}
\end{block}



\end{frame}

\section{Other} 
\begin{frame} {Other}
=:= \\
\textless:!\textless
\end{frame}

\end{document}