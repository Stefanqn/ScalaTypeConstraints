\documentclass[11pt]{beamer}
%\hypersetup{hidelinks}
\usetheme{Montpellier}
%\usetheme{Pittsburgh}

\usepackage[utf8]{inputenc}
\usepackage{scala2}
\usepackage{amsmath}
\usepackage{amsfonts}
\usepackage{amssymb}
\usepackage{graphicx}
\usepackage{listings}
% --------------------------------------------------------------------------

% --------------------------------------------------------------------------

\author{Stefan}
%\title{\textless:\textless}
\title{Generalized type constraints}
%\setbeamercovered{transparent} 
%\setbeamertemplate{navigation symbols}{} 
%\logo{} 
%\institute{} 
\date{} 
\subject{sad-with-a-hat} 
\begin{document}


%% Title
\begin{frame}
\titlepage
Based on \href{http://blog.bruchez.name/2015/11/generalized-type-constraints-in-scala.html}{Generalized type constraints in Scala}
\end{frame}

%% ToC
\begin{frame} 
	\frametitle{Table of content} 
	\tableofcontents 
\end{frame}

\section{Introduction} 
\subsection{Usage}

\begin{frame}[fragile]
\frametitle{How is it useful?}
\begin{itemize}
\item on Option
\begin{lstlisting} 
def flatten[B](implicit ev: A <:< Option[B]): Option[B]
\end{lstlisting} %\pause

\item on Traversable
\begin{lstlisting} 
def toMap[K, V](implicit ev: A <:< (K, V)): Map[K, V]
\end{lstlisting} %\pause

\item on Try
\begin{lstlisting} 
def flatten[U](implicit ev: T <:< Try[U]): Try[U]
\end{lstlisting} 
\end{itemize}
\end{frame}

\begin{frame}{What do they have in common?}
\begin{itemize}
\item implicit parameter list, with a single parameter called ev
\item type of this parameter is of the form \lstinline{Type1 <:< Type2}
\end{itemize}
\end{frame}


\subsection{Meaning}
\begin{frame}[fragile]{Meaning}
\lstinline {Type1 <:< Type2} \newline\newline
\begin{block}{Meaning} %definition, theorem, lemma, proof, corollary, or example
\textit{Make sure that Type1 is a subtype of Type2, or else report an error.}
\end{block}
\end{frame}

\begin{frame}[fragile]

\begin{block}{Example 1} 
\begin{lstlisting} 
scala> val oo: Option[Option[Int]] = Some(Some(42))
oo: Option[Option[Int]] = Some(Some(42))

scala> oo.flatten
res1: Option[Int] = Some(42)
\end{lstlisting} 
\end{block}

\begin{block}{Example 2}
\begin{lstlisting} 
scala> val oi: Option[Int] = Some(42)
oi: Option[Int] = Some(42)

scala> oi.flatten
<console>:21: error: Cannot prove that Int <:< Option[B].
       oi.flatten
\end{lstlisting} 
\end{block} 
\end{frame}


\section{Type bound usage} 
\subsection{Can’t we just use type bounds?}
\begin{frame}[fragile] {Plain example}
\begin{lstlisting} 
scala> def tuple[T, U](t: T, u: U) = (t, u)
scala> tuple("Lincoln", 42)
res1: (String, Int) = (Lincoln,42)
\end{lstlisting}
\end{frame}

\begin{frame}[fragile] {Example with upper bound}
\begin{lstlisting} 
scala> def tupleIfSubtype[T <: U, U](t: T, u: U) = (t, u)
scala> tupleIfSubtype("Lincoln", 42)
\end{lstlisting} 
\begin{uncoverenv}<2> 
\begin{lstlisting} 
res2: (String, Any) = (Lincoln,42)
\end{lstlisting} 
\end{uncoverenv}
\end{frame}

\subsection{Constraint system}
\begin{frame}[fragile] {Puzzler}
\begin{block}{Why?}
\textit{type inference solves a constraint system}
\end{block}\pause

\begin{block}{so given}
\begin{lstlisting} 
scala> def tupleIfSubtype[T <: U, U](t: T, u: U) = (t, u)
scala> tupleIfSubtype("Lincoln", 42)
res2: (String, Any) = (Lincoln,42)
\end{lstlisting} 
\end{block}

\begin{block}{the constraints are satisfied with}
\begin{itemize}
\item String is a String of course
\item Int is a subtype of Any
\item String is also a subtype of Any
\end{itemize}
\end{block}
\end{frame}



\section{How does it work?}
\subsection{Simplified}
\begin{frame}[fragile] {Simplified implementation}
\begin{lstlisting} 
trait <::<[-From, +To]

object <::< {
  implicit def conforms[A]: A <::< A = new <::<[A,A]{}
}
\end{lstlisting} \pause

\begin{block}{e.g.}
\begin{lstlisting} 
def tupleIfSubtype[T, U](t: T, u: U)(implicit ev: T <::< U) = (t, u)

tupleIfSubtype(new Apple(), new Fruit())(<::<.conforms[Fruit])
tupleIfSubtype(new Apple(), new Fruit())(<::<.conforms[Apple])
\end{lstlisting} 
\end{block}
\end{frame}

\subsection{Explained}
\begin{frame}{Why does it work?}
\begin{itemize}
\item Variance of the type parameters of 
\lstinline{trait <::<[-From, +To]} is similar to a function: require less, provide more: $F \Rightarrow T$ 
\item $\rightarrow$ compiler tries to find a type \lstinline{<::<[A, A]} that conforms to e.g. \lstinline{<::<[Apple, Fruit]} 
\end{itemize}
\end{frame}

\begin{frame}[fragile] {What constraints must be satisfied?}
\begin{itemize}
\item $F \Rightarrow T$ is required, what are the constraints to pass
\item $A \Rightarrow A$? \pause
\item Due to variance \lstinline{trait <::<[-F, +T]}:
\begin{itemize}
\item A must be a supertype of F: \lstinline{A >: F} or \lstinline{F <: A}
\item A must be a subtype of T: \lstinline{A <: T} 
\end{itemize}\pause
\item $\rightarrow$ \lstinline{F <: A <: T}  
\item $\rightarrow$ \lstinline{F <: T}  
\end{itemize}
\end{frame}

\subsection{Implementation}
\begin{frame}{Real implementation}
\begin{itemize}
\item adds a nice error message: \newline
\lstinline{\@implicitNotFound(msg = "Cannot prove that $\{From\} <:< $\{To\}.")}
\item extends function: 
\lstinline{class <:<[-From, +To] extends (From => To)}
\item uses a singleton value: \newline
\lstinline{val singleton_<:< =}
\lstinline{new <:<[Any,Any] \{ def apply(x: Any): Any = x \}}
\item and typecast it:\\
\lstinline{singleton_<:<.asInstanceOf[A <:< A]}
\item evidence also used as conversion $From \Rightarrow To$, e.g. in Option: \newline
\lstinline{def flatten[B](implicit ev: A <:< Option[B]): Option[B] =
if (isEmpty) None else ev(this.get)}
\end{itemize}
\end{frame}

\section{Other} 
\begin{frame}[fragile] {Other}
\begin{itemize}
\item alternative: implicit class or type class $\rightarrow$ more overhead
\item exact type \lstinline{=:=[From, To]} vs \lstinline{<:<[-From, +To]} \pause
\item not subtype \textless:!\textless
\begin{lstlisting} 
def unexpected : Nothing = sys.error("Unexpected invocation")
@scala.annotation.implicitNotFound("${A} must not be a subtype of ${B}")
trait <:!<[A, B] extends Serializable
implicit def nsub[A, B] : A <:!< B = new <:!<[A, B] {}
implicit def nsubAmbig1[A, B >: A] : A <:!< B = unexpected
implicit def nsubAmbig2[A, B >: A] : A <:!< B = unexpected
\end{lstlisting} 
\end{itemize}
\end{frame}

\end{document}